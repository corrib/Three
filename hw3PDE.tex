\documentclass[12pt,a4paper,oneside,draft]{article}
\title{Homework Three}
\author{Corrina Black}
\date{April 2014}
\linespread{1.1}
\usepackage{amsmath}
\usepackage{color}

\begin{document}
\noindent\textsc{\Large\textbf{Homework Three}}
\newline
\newline
\normalsize  
Choose two famous partial differential equations. Look up their history and application, and write a paragraph about each describing what makes them unique and important. Include at least one source for each, and cite it in MLA format. (3pts each)
\begin{enumerate}
     \large\item\textbf{Wave Equation $\frac{\mathrm d^2 u}{\mathrm d t^2} - c^2 \frac{\mathrm d^2 u}{\mathrm d x^2} = 0$}
	 \newline
     \begin{normalsize}
     The wave equation is a partial differential equation for describing waves such as sound waves, light waves, and water waves. Jean le Rond d'Alembert discovered the one-dimensional wave equation in 1746 when exploring the problem of vibrating strings such as that of a musical instrument. The wave equation is used in fields such as acoustics and fluid dynamics. (Wikipedia, Wave equation, pars. 1-2)\footnote{\tiny\raggedright Wikipedia. "Wave Equation." Web. 28 April 2014 $<http://en.wikipedia.org/wiki/Wave_equation>$.} The wave equation is important because the behavior of waves generalizes to the way sound works, how earthquakes happen, and the behavior of the ocean. (Business Insider, The wave equation)\footnote{\tiny\raggedright Business Insider. "The 17 Equations That Changed The World." Web. 28 April 2014 $<http://www.businessinsider.com/the-17-equations-that-changed-the-world-2012-7?op=1>$.}
     \end{normalsize}
     \large\item\textbf{Heat Equation [time dependent] $\frac{\mathrm d u}{\mathrm d t} - \alpha \frac{\mathrm d^2 u}{\mathrm d x^2} = 0$}
	 \newline
     \begin{normalsize}
     In 1822, Jean Baptiste Joseph Fourier published his work on heat flow in Théorie analytique de la chaleur (The Analytic Theory of Heat). There were three important contributions in this work, one of which was Fourier's proposal of his partial differential equation for conductive diffusion of heat (the heat equation). This equation is now taught to every student of mathematical physics.(Wikipedia, Joseph Fourier, The Analytic Theory of Heat)\footnote{\tiny\raggedright Wikipedia. "Joseph Fourier." Web. 28 April 2014 $<http://en.wikipedia.org/wiki/Joseph_Fourier>$.} The heat equation is important because it describes the distribution of heat (or variation in temperature) in a given region over time. (Wikipedia, Heat equation, par 1)\footnote{\tiny\raggedright Wikipedia. "Heat equation." Web. 28 April 2014 $<http://en.wikipedia.org/wiki/Heat_equation>$.}
     \end{normalsize}
\end{enumerate}
\end{document}